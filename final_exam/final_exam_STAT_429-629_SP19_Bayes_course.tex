% Exam Template for UMTYMP and Math Department courses
% http://www-users.math.umn.edu/~rogness/math8001/handouts/math_exam_template.tex
% Using Philip Hirschhorn's exam.cls: http://www-math.mit.edu/~psh/#ExamCls
%
% run pdflatex on a finished exam at least three times to do the grading table on front page.
%
%%%%%%%%%%%%%%%%%%%%%%%%%%%%%%%%%%%%%%%%%%%%%%%%%%%%%%%%%%%%%%%%%%%%%%%%%%%%%%%%%%%%%%%%%%%%%%

% Modified by AG Schissler

% These lines can probably stay unchanged, although you can remove the last
% two packages if you're not making pictures with tikz.
\documentclass[11pt]{exam}
\RequirePackage{amssymb, amsfonts, amsmath, latexsym, verbatim, xspace, setspace}
\RequirePackage{tikz, pgflibraryplotmarks}

% By default LaTeX uses large margins.  This doesn't work well on exams; problems
% end up in the "middle" of the page, reducing the amount of space for students
% to work on them.
\usepackage[margin=1in]{geometry}


% Here's where you edit the Class, Exam, Date, etc.
\newcommand{\class}{MATH/STAT 429/629}
\newcommand{\term}{Spring 2019}
\newcommand{\examnum}{Final Exam}
\newcommand{\examdate}{05/15/19}
\newcommand{\timelimit}{110 Minutes}

% For an exam, single spacing is most appropriate
\singlespacing
% \onehalfspacing
% \doublespacing

% For an exam, we generally want to turn off paragraph indentation
\parindent 0ex

\begin{document} 

% These commands set up the running header on the top of the exam pages
% \pagestyle{head}
\firstpageheader{}{}{}
\runningheader{\class}{\examnum\ - Page \thepage\ of \numpages}{\examdate}
\runningheadrule

% these commands set up the running footer
\firstpagefooter{}{}{}
\runningfooter{}{}{Points earned out of a possible \pointsonpage{\thepage} points on the page:~\makebox[1in]{\hrulefill}}
\extrafootheight{-0.25in}


\begin{flushright}
\begin{tabular}{p{2.8in} r l}
\textbf{\class} & \textbf{Name (Print):} & \makebox[2in]{\hrulefill}\\
\textbf{\term} &&\\
\textbf{\examnum} &&\\
\textbf{\examdate} &&\\
\textbf{Time Limit: \timelimit} & Signature & \makebox[2in]{\hrulefill}
\end{tabular}\\
\end{flushright}
\rule[1ex]{\textwidth}{.1pt}


This exam contains \numpages\ pages (including this cover page) and
\numquestions\ problems.  Check to see if any pages are missing.  Enter
all requested information on the top of this page, and put your initials
on the top of every page, in case the pages become separated.\\

You may use two pages of hand-written notes (front and back), scratch paper, and a scientific calculator on this exam.\\

You are required to show your work on each free-response problem on this exam (you are not required to show work on multiple choice).  The following rules apply:\\

\begin{minipage}[t]{3.7in}
\vspace{0pt}
\begin{itemize}

\item \textbf{Write the letter corresponding to your choice on the blank} to answer multiple choice questions.

\item \textbf{Organize your work} for free-response problems, in a reasonably neat and coherent way, in
the space provided. Work scattered all over the page without a clear ordering will 
receive very little credit. If you need extra space write the \textbf{solution or answer} on the page and attach supporting work by stapling extra pages to the exam.

\item \textbf{Mysterious or unsupported answers will not receive full
credit}.  A correct answer, unsupported by calculations, explanation,
or algebraic work will receive no credit; an incorrect answer supported
by substantially correct calculations and explanations might still receive
partial credit.

\item Please do not write in the table to the right or on the footer at the bottom each page.
\end{itemize}


\end{minipage}
\hfill
\begin{minipage}[t]{2.3in}
\vspace{0pt}
%\cellwidth{3em}
\gradetablestretch{2}
\vqword{Problem}
\addpoints % required here by exam.cls, even though questions haven't started yet.	
%\gradetable[v]%[pages]  % Use [pages] to have grading table by page instead of question
\gradetable[v][pages]%[pages]  % Use [pages] to have grading table by page instead of question

\end{minipage}
\newpage % End of cover page

%%%%%%%%%%%%%%%%%%%%%%%%%%%%%%%%%%%%%%%%%%%%%%%%%%%%%%%%%%%%%%%%%%%%%%%%%%%%%%%%%%%%%
%
% See http://www-math.mit.edu/~psh/#ExamCls for full documentation, but the questions
% below give an idea of how to write questions [with parts] and have the points
% tracked automatically on the cover page.
%
%
%%%%%%%%%%%%%%%%%%%%%%%%%%%%%%%%%%%%%%%%%%%%%%%%%%%%%%%%%%%%%%%%%%%%%%%%%%%%%%%%%%%%%

\begin{questions}

  \section{Multiple-choice questions}
  % Multiple choice 25 questions
% Q1 1.1
  \addpoints
  \question[2]
  Solve the system by using elementary row operations on the equations. \newline Follow the elimination procedure. The order of answer choices is $(x_{1},x_{2}).$\newline
  $ \begin{array}{ccc}\phantom{3}x_{1}+3x_{2} & = & -1 \\ 4x_{1}+5x_{2} & = & -18 \\ \end{array} $

\begin{oneparchoices}
\choice $(-7,2)$ %this one
\choice $(0,0)$
\choice $(-1,-18)$
\choice $(-3,3)$
\end{oneparchoices}
\answerline

% Q2
\addpoints
\question[2]
Find the general solution of the system whose \textbf{augmented} matrix is given below. \newline $\left[ \begin{array}{cccc} 3 & -5 & 2 & 0 \\ 9 & -15 & 6 & 0 \\ 12 & -20 & 8 & 0 \\ \end{array}  \right]$

\begin{oneparchoices}
\choice $\left\{ \begin{array}{l} x_{1} = -5 x_{2} \\ x_{2} = 3x_{3} \\ x_{3} \text{ is free} \\ \end{array} \right.$
\choice $\left\{ \begin{array}{l} x_{1} = \frac{5}{3} x_{2} - \frac{2}{3} x_{3}\\ x_{2} \text{ is free} \\ x_{3} \text{ is free} \\ \end{array} \right.$ %this one
\choice $\left\{ \begin{array}{l} x_{1} = 5 \\ x_{2} = -3 \\ x_{3}=2 \\ \end{array} \right.$
\choice No solution(s).
\end{oneparchoices}
\answerline

% Q3
\addpoints
\question[2]
Compute $\mathbf{u}-2\mathbf{v}$ when $\mathbf{u} = \left[\begin{array}{c} -2 \\ \phantom{-}9 \\ \end{array}\right]$ and $\mathbf{v} = \left[\begin{array}{c} -4 \\ \phantom{-}3 \\ \end{array}\right]$.

\begin{oneparchoices}
\choice $\left[\begin{array}{c} 6 \\ 3 \\ \end{array}\right]$ %this one
\choice $\left[\begin{array}{c} -2 \\ 0 \\ \end{array}\right]$
\choice $\left[\begin{array}{c} -6 \\ -3 \\ \end{array}\right]$
\choice $\left[\begin{array}{c} 3 \\ 4 \\ \end{array}\right]$
\end{oneparchoices}
\answerline

% Q4 
\addpoints
\question[2]
Use the definition of $A\mathbf{x}$ to write the vector equation as a matrix equation. \newline
$x_{1}\left[\begin{array}{c}-8 \\ \phantom{-}7 \\ -5 \\ \phantom{-}6 \end{array}\right] + x_{2}\left[\begin{array}{c}-9 \\ -4 \\ -8 \\ \phantom{-}3\end{array}\right] + x_{3}\left[\begin{array}{c}-6 \\ \phantom{-}0 \\ \phantom{-}6 \\ \phantom{-}9 \end{array}\right] = \left[\begin{array}{c}1 \\ 8 \\ 7 \\ 5 \end{array}\right]$ \newline
\begin{oneparchoices}
\choice $\left[\begin{array}{ccc}-8 & -9 & -6 \\ \phantom{-}7 & -4 & \phantom{-}0 \\ -5 & -8 & \phantom{-}6 \\ \phantom{-}6 & \phantom{-}3 & \phantom{-}9 \\ \end{array}\right] \left[\begin{array}{c}x_{1}\\x_{2}\\x_{3}\end{array}\right]= \left[\begin{array}{c}1\\8\\7\\5\end{array}\right]$ %this one
\choice $\left[\begin{array}{ccc}-8 & -9 & -6 \\ \phantom{-}7 & -4 & \phantom{-}0 \\ -5 & -8 & \phantom{-}6 \\ \phantom{-}6 & \phantom{-}3 & \phantom{-}9 \\ \end{array}\right] \left[\begin{array}{c}x_{1}\\x_{2}\\x_{3}\\x_{4}\end{array}\right]= \left[\begin{array}{c}1\\8\\7\\5\end{array}\right]$\\
\choice $\left[\begin{array}{ccc}-8 & -9 & -6 \\ \phantom{-}7 & -4 & \phantom{-}0 \\ -5 & -8 & \phantom{-}6 \\ \phantom{-}6 & \phantom{-}3 & \phantom{-}9 \\ \end{array}\right] \left[\begin{array}{c}x_{1}\\x_{2}\end{array}\right]= \left[\begin{array}{c}1\\8\\7\\5\end{array}\right]$
\choice The matrices are not compatible. 
\end{oneparchoices}
\answerline

% Q5
\addpoints
\question[2]
Determine the nature of solutions to the following system. \newline$ \begin{array}{cccc}6x_{1}-3x_{2} +15x_{3}& = & 0 \\ -6x_{1}-9x_{2} -6x_{3}& = & 0 \\ 12x_{1}+6x_{2} +21x_{3}& = & 0 \\ \end{array} $

\begin{oneparchoices}
\choice It is impossible to determine.
\choice The system has only a trivial solution.\\
\choice The system has a unique, nontrivial solution.
\choice The system has infinite solutions.  %this one
\end{oneparchoices}
\answerline

% Q6
\addpoints
\question[2]
Determine by inspection whether the vectors are linearly independent.\newline
$\left[\begin{array}{c} 4 \\ 1 \\ \end{array}\right], \left[\begin{array}{c} 2 \\ 9 \\ \end{array}\right], \left[\begin{array}{c} 1 \\ 5 \\ \end{array}\right], \left[\begin{array}{c} -1 \\ \phantom{-}8 \\ \end{array}\right]$

\begin{choices}
\choice The set is linearly independent because at least one of the vectors is a multiple of another vector.
\choice The set is linearly dependent because at least one of the vectors is a multiple of another vector.
\choice The set is linearly dependent because there are four vectors but only two entries in each vector. %this one
\choice The set is linearly independent because there are four vectors in the set but only two entries in each vector.
\end{choices}
\answerline

% Q7
  \addpoints
  \question[2]
Let $T: \Re^{2}\rightarrow \Re^{2}$ be a linear transformation that maps $\mathbf{u}=\left[\begin{array}{c} 2 \\ 3 \\ \end{array}\right]$ into $\left[\begin{array}{c} 6 \\ 1 \\ \end{array}\right]$ and maps $\mathbf{v}=\left[\begin{array}{c} 3 \\ 3 \\ \end{array}\right]$ into $\left[\begin{array}{c} -1 \\ \phantom{-}3 \\ \end{array}\right]$. Use the fact that $T$ is linear to find the image under T of $3\mathbf{u}+2\mathbf{v}$. 

\begin{oneparchoices}
\choice $\left[\begin{array}{c} -16 \\ -9 \\ \end{array}\right]$
\choice $\left[\begin{array}{c} 16 \\ 9 \\ \end{array}\right]$ %this one
\choice $\left[\begin{array}{c} -9 \\ 16 \\ \end{array}\right]$
\choice $\left[\begin{array}{c} 9 \\ 16 \\ \end{array}\right]$
\end{oneparchoices}
\answerline

% Q8
\addpoints
\question[2]
If a matrix $A$ is $8\times 5$ and the product $AB$ is $8 \times 3$, what is the size of​ $B$?

\begin{oneparchoices}
\choice $3 \times 5$
\choice $5 \times 5$
\choice $5 \times 3$ %this one
\choice $3 \times 3$
\end{oneparchoices}
\answerline

% Q9
\addpoints
\question[2]
Find the inverse of the matrix. $\left[\begin{array}{cc}\phantom{-}9 & \phantom{-}3 \\-8 & -3 \\ \end{array}\right]$

\begin{oneparchoices}
\choice The matrix is not invertible.
\choice $\left[\begin{array}{cc}\phantom{-}1 & -1 \\ \phantom{-}\frac{8}{3} & -3 \\ \end{array}\right]$
\choice $\left[\begin{array}{cc}\phantom{-}3 & \phantom{-}3 \\-8 & -9 \\ \end{array}\right]$
\choice $\left[\begin{array}{cc}\phantom{-}1 & \phantom{-}1 \\-\frac{8}{3} & -3 \\ \end{array}\right]$ %this one
\end{oneparchoices}
\answerline

% Q10
\addpoints
\question[2]
Determine if the matrix below is invertible. Use as few calculations as possible. Justify your answer. \newline
$\left[\begin{array}{ccc}\phantom{-}5 & \phantom{-}0 & \phantom{-}0 \\ -4 & -5 & \phantom{-}0 \\ \phantom{-}7 & \phantom{-}4 & -2 \\ \end{array}\right]$

\begin{choices}
\choice The matrix is not invertible. If the given matrix is​ $A$, the equation $A\mathbf{x}=\mathbf{b}$ has no solution for some $\mathbf{b}$ in $\Re^{3}$.
\choice The matrix is invertible. The given matrix has three pivot positions. %this one
\choice The matrix is invertible. If the given matrix is $A$, there is a $3 \times 3$ matrix $C$ such that $CI=A$.
\choice The matrix is not invertible. The given matrix has two pivot positions.
\end{choices}
\answerline

% Q11
\addpoints
\question[2] Compute the determinant of the following matrix. \newline
$\left[\begin{array}{ccc}1 & w & 0 \\ 0 & 1 & 0 \\ 0 & 0 & 1 \\ \end{array}\right]$.

\begin{oneparchoices}
\choice $w$
\choice 3
\choice 1 %this one 
\choice Cannot say.
\end{oneparchoices}
\answerline

% Q12
\addpoints
\question[2]
Find the determinant below, where $\left|\begin{array}{ccc}a & b & c \\ d & e & f \\ g & h & i \\ \end{array}\right|=4$. \newline
$\left|\begin{array}{ccc}a & b & c \\ d & e & f \\ 5g & 5h & 5i \\ \end{array}\right|$

\begin{oneparchoices}
\choice 4
\choice 20 %this one
\choice -4
\choice Cannot say
\end{oneparchoices}
\answerline

% Q13
\addpoints
\question[2]
Let $H$ be the set of all vectors of the form $\left[\begin{array}{c}5t \\ 2t \\ 4t \end{array}\right]$. Find a vector $\mathbf{v}$ in $\Re^{3}$ such that $H=\text{Span}\{{\mathbf{v}}\}$.

\begin{oneparchoices}
\choice $\mathbf{v} = \left[\begin{array}{c}0 \\ 0 \\ 0 \end{array}\right]$
\choice $\mathbf{v} = \left[\begin{array}{c}5 \\ 2 \\ 4 \end{array}\right]$ %this one
\choice $\mathbf{v} = \left[\begin{array}{c}1 \\ 1 \\ 1 \end{array}\right]$
\choice No such $\mathbf{v}$ exists.
\end{oneparchoices}
\answerline

% Q14
\addpoints
\question[2]
Determine if $\mathbf{w}= \left[\begin{array}{c}2 \\ 5 \\ 1 \end{array}\right]$ is in $\text{Nul }A$, where $A=\left[\begin{array}{ccc}\phantom{-}3 & -5 & 4 \\ \phantom{-}5 & -3 & 1 \\ -4 & \phantom{-}4 & 3 \\ \end{array}\right]$

\begin{oneparchoices}
\choice No, because $A\mathbf{w}=\left[\begin{array}{c}-15 \\ -4 \\ 15 \end{array}\right]$ %this one
\choice No, the Nul $A$ is empty. \\
\choice Yes, because $A\mathbf{w}=\left[\begin{array}{c}0 \\ 0 \\ 0 \end{array}\right]$
\choice Cannot say.
\end{oneparchoices}
\answerline

% Q15
\addpoints
\question[2]
Determine if the set of vectors shown to the right is a basis for $\Re^{3}$.\newline
$\left\{\left[\begin{array}{c} \phantom{-}2 \\ \phantom{-}3 \\ -9 \end{array}\right], \left[\begin{array}{c} -5 \\ \phantom{-}4 \\ 12 \\ \end{array}\right]\right\}$.

\begin{oneparchoices}
\choice No, the set does not span $\Re^{3}$. %this one
\choice Yes, the set is a basis for $\Re^{3}$.\\
\choice No, the set is linearly dependent.
\choice Cannot say.
\end{oneparchoices}
\answerline

% Q16
\addpoints
\question[2]
State the dimension for the subspace below.\newline
$\left\{\left[\begin{array}{c} p-9q\\5p+8r\\-9q+8r\\-6p+12r\\ \end{array}\right]:p,q,r \text{ in } \Re \right\}$.

\begin{oneparchoices}
\choice 0
\choice 2
\choice 3 %this one
\choice 4
\end{oneparchoices}
\answerline

% Q17
\addpoints
\question[2]
If the null space of a $6 \times 8$ matrix $A$ is 4​-dimensional, what is the dimension of the column space of​ $A$?

\begin{oneparchoices}
\choice 4 %this one
\choice 6
\choice 8
\choice 0
\end{oneparchoices}
\answerline

% Q18
\addpoints
\question[2]
Is $\mathbf{v}=\left[\begin{array}{c}\phantom{-}3 \\ \phantom{-}1 \\ -2 \\ \end{array}\right]$ an eigenvalue of $A=\left[\begin{array}{ccc} 3 & 6 & 7 \\ 3 & 2 & 7 \\ 5 & 6 & 4 \\ \end{array}\right]$? If so, find the eigenvalue $\lambda$.

\begin{oneparchoices}
\choice Yes, and $\lambda=2$.
\choice No, $\mathbf{v}$ is not an eigenvalue of $A$. \\ %this one
\choice Yes, and $\lambda=-2$.
\choice Yes, and $\lambda=0$.
\end{oneparchoices}
\answerline

% Q19
\addpoints
\question[2]
Find the eigenvalues of $\left[\begin{array}{cc}4 & 2 \\ 2 & 4 \\ \end{array}\right]$.

\begin{oneparchoices}
\choice 3,~3
\choice 2,~6 %this one
\choice -3,~-3
\choice 2,~4
\end{oneparchoices}
\answerline

% Q20
\addpoints
\question[2]
Identify a nonzero $2 \times 2$ matrix that is invertible but not diagonalizable.

\begin{oneparchoices}
\choice $\left[\begin{array}{cc}1 & 0 \\ 0 & 0 \\ \end{array}\right]$
\choice $\left[\begin{array}{cc}1 & 0 \\ 0 & 1 \\ \end{array}\right]$
\choice $\left[\begin{array}{cc}1 & 1 \\ 0 & 0 \\ \end{array}\right]$
\choice $\left[\begin{array}{cc}1 & 1 \\ 0 & 1 \\ \end{array}\right]$ %this one
\end{oneparchoices}
\answerline

\newpage
\addpoints
\question[2]
Compute $||\mathbf{w}||$ using $\mathbf{w}=\left[\begin{array}{c}\phantom{-}2 \\ -4 \\ -2 \\\end{array}\right]$.

\begin{oneparchoices}
\choice $2\sqrt{6}$ %this one
\choice 24
\choice -16
\choice 1
\end{oneparchoices}
\answerline

% Q22
\addpoints
\question[2]
Determine whether the set of vectors is orthogonal. \newline
$\left[\begin{array}{c} 0 \\ 0 \\ 0 \\ \end{array}\right], \left[\begin{array}{c} \phantom{-}2 \\ -5 \\ -9 \\ \end{array}\right], \left[\begin{array}{c} \phantom{-}4 \\ -2 \\ \phantom{-}2 \\ \end{array}\right]$

\begin{choices}
\choice No, the zero vector is in the set.
\choice Yes, each pair of vectors are orthogonal to each other. %this one
\choice No, the second and third vectors are not orthogonal.
\choice Cannot say.
\end{choices}
\answerline

% Q23
\addpoints
\question[2]
Find the orthogonal projection of $\mathbf{y}$ onto Span$\{\mathbf{u_{1},u_{2}}\}$.\newline
$\mathbf{y} = \left[\begin{array}{c} \phantom{-}3 \\ \phantom{-}4 \\ -4 \\ \end{array}\right], \mathbf{u}_{1} = \left[\begin{array}{c} 2 \\ 3 \\ 0 \\ \end{array}\right], \mathbf{u}_{2}= \left[\begin{array}{c} -3 \\ \phantom{-}2 \\ \phantom{-}0 \\ \end{array}\right]$

\begin{oneparchoices}
\choice $\left[\begin{array}{c} -3 \\ 2 \\ 0 \\ \end{array}\right]$
\choice $\left[\begin{array}{c} 0 \\ 0 \\ 0 \\ \end{array}\right]$
\choice $\left[\begin{array}{c} 3 \\ 4 \\ 0 \\ \end{array}\right]$ %this one
\choice $\left[\begin{array}{c} 3 \\ 4 \\ 4 \\ \end{array}\right]$
\end{oneparchoices}
\answerline

% Q24
\addpoints
\question[2]
Determine if the matrix $A= \left[\begin{array}{cc}-5 & \phantom{-}9 \\ \phantom{-}4 & -9 \\ \end{array}\right]$ is symmetric.

\begin{choices}
\choice The matrix is not symmetric, since $A^{T}=\left[\begin{array}{cc}-5 & \phantom{-}4 \\ \phantom{-}9 & -9 \\ \end{array}\right]$. %this one
\choice The matrix is symmetric, since $A^{T}=\left[\begin{array}{cc}-5 & \phantom{-}9 \\ \phantom{-}4 & -9 \\ \end{array}\right]$.
\choice The matrix is not symmetric, since the matrix is not invertible.
\choice Cannot say.
\end{choices}
\answerline

% Q25
\addpoints
\question[2]
Compute the quadratic form $\mathbf{x}^{T}A\mathbf{x}$ for $A=\left[\begin{array}{cc}2 & \frac{1}{3} \\ \frac{1}{3} & 1 \\ \end{array}\right]$ for $\mathbf{x}=\left[\begin{array}{c} x_{1} \\ x_{2} \\ \end{array}\right]$.

\begin{oneparchoices}
\choice 161
\choice $2x_{1}^{2}+\frac{1}{9}x_{1}x_{2}+x_{2}^{2}$
\choice $4x_{1}^{2}+\frac{2}{3}x_{1}x_{2}+x_{2}^{2}$
\choice $2x_{1}^{2}+\frac{2}{3}x_{1}x_{2}+x_{2}^{2}$ %this one
\end{oneparchoices}
\answerline

%%% Free response
\section{Free-response questions}
\newpage

% Q26

\addpoints
\question[5] Let $\mathbf{v}_{1}=\left[ \begin{array}{c} -1 \\ -3 \\ \phantom{-}2 \end{array} \right]$, $\mathbf{v}_{2}=\left[ \begin{array}{c} 3 \\ 3 \\ 1 \end{array} \right]$, and $\mathbf{v}_{3}=\left[ \begin{array}{c} 1 \\ -15 \\ 19 \end{array} \right]$. \\ Determine if the set $\{\mathbf{v}_{1}, \mathbf{v}_{2}, \mathbf{v}_{3}\}$ is linearly independent.
\vspace{\stretch{1}}
% \vfill

% Q27
\question[5] Let $\{\mathbf{x}_{1}, \mathbf{x}_{2}\}$ be a basis for a subspace $W$, where $\mathbf{x}_{1}=\left[ \begin{array}{c} \phantom{-}3 \\ \phantom{-}1 \\ -1 \\ \phantom{-}2 \end{array} \right]$ and $\mathbf{x}_{2}=\left[ \begin{array}{c} \phantom{-}2 \\ -1 \\ -1 \\ \phantom{-}1 \end{array} \right]$. Use the Gram-Schmidt process to construct an orthonormal basis for $W$.
\vspace{\stretch{2}}
% \vfill

\newpage
% Q28
\question[5] Orthogonally diagonalize the matrix $A= \left[\begin{array}{cc}-3 & 3 \\ \phantom{-}3 & 5 \\ \end{array}\right]$ by giving an orthogonal matrix $P$ and a diagonal matrix $D$ such that $A=PDP^{T}$.
\vspace{\stretch{2}}
% Q29 2 parts

\addpoints
\question Consider $A=\left[\begin{array}{ccccc}\phantom{-}1 & \phantom{-}2 & \phantom{-}0 & -1 & \phantom{-}20 \\ \phantom{-}1 & \phantom{-}1 & -1 & \phantom{-}1 & \phantom{-}7 \\ -2 & -1 & \phantom{-}3 & -3 & -5\end{array}\right]$. 
\begin{parts}
  \part[5] What is the rank of $A$? (Hint:~find an echelon form)
  \vspace{\stretch{1}}
%\vspace{\stretch{1}}
  \part[5] What is the dimension of the null space of $A$? (Write a short justification).
\vspace{0.5in}
%\part[5] Find a basis for the null space of $A$.
%\vspace{\stretch{2}}
\end{parts}
\newpage
% Q30 3 parts

\addpoints
\question Find the singular value decomposition $U\Sigma V^{T}$ for $A=\left[ \begin{array}{cc} \phantom{-}1 & 1 \\ \phantom{-}0 & 1 \\ -1 & 1 \end{array} \right]$ by completing the following steps:
\begin{parts}
\part[5] Find an orthogonal diagonalization of $A^{T}A$.
\vspace{\stretch{2}}
\part[5] Set up $V$ by using the normalized eigenvectors from above and $\Sigma$  by using the singular values of $A$ and zeros where appropriate.
\vspace{\stretch{1}}
\part[5] Construct $U$ (hint:~$\mathbf{u}_{i}=\frac{1}{\sigma_{i}}A\mathbf{v}_{i}$).
\vspace{\stretch{2}}
\end{parts}

% Q31 2 parts
\newpage
\addpoints
\question Consider $A=\left[ \begin{array}{cc} -1 & 2 \\ 2 & -3 \\ -1 & 3 \end{array} \right]$ and $\mathbf{b}=\left[\begin{array}{c} 4 \\ 1 \\ 2 \\ \end{array}\right]$.
\begin{parts}
\part[5] Does the equation $A\mathbf{x}=\mathbf{b}$ have a solution? (Justify with a calculation).
\vspace{\stretch{1}}
\part[5] If not, find the least squares solution $\mathbf{\hat{x}}$ by solving the normal equations \\$A^{T}A\mathbf{\hat{x}}=A^{T}\mathbf{b}$.
\vspace{\stretch{2}}
%\part[5] Find a basis for the null space of $A$.
%\vspace{\stretch{2}}
\end{parts}


% % If you want the total number of points for a question displayed at the top,
% % as well as the number of points for each part, then you must turn off the point-counter
% % or they will be double counted.
% 
% \addpoints
% \question[10] Consider the function $f(x)=x^3$.
% \noaddpoints % If you remove this line, the grading table will show 20 points for this problem.
% \begin{parts}
% \part[5] Find $f'(x)$ using the limit definition of derivative.
% \vspace{4.5in}
% \part[5] Find the line tangent to the graph of $y=f(x)$ at the point where $x=2$.
% \end{parts}
 


\end{questions}
\end{document}