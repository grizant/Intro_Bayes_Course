%A simple LaTeX syllabus template by Luke Smith.
%I believe this is a variation of a template I got by Mike Hammond...?

%\documentclass[11pt,twocolumn]{article}
\documentclass[11pt,onecolumn]{article}
\usepackage{hyperref,color}
\usepackage[margin=1in]{geometry}
\usepackage{titlesec}
\usepackage{longtable}
\usepackage{gb4e}
%\usepackage[margin=.75in]{geometry}
\usepackage{textcomp}
\pagenumbering{gobble}

%paragraph formatting
\setlength{\parindent}{0pt}
\setlength{\parskip}{7pt}

\hypersetup{
    colorlinks=true,
    linkcolor=blue,
    filecolor=magenta,      
    urlcolor=blue,
}


\title{Introduction to Applied Bayesian Analysis (STAT 446/646) \\ 400/600 cross-listed course questionnaire}
\renewcommand{\familydefault}{\sfdefault}

\begin{document}
\date{}
\maketitle

\vspace{-0.75in}

\begin{enumerate}
\item \emph{How will graduate students achieve deeper understanding of the material presented to the combined group?}

  Modeling/computational skill is obtained mostly through practice and effort. Graduate students will be expected to complete more exercises and exercises of greater difficulty. The selected text offers problems of three scaffolds - ``easy'', ``medium'', and ``hard''. The graduate students will often complete ``hard'' problems while undergraduate students will complete mostly ``easy'' and ``medium''. Graduate students will also be expected to master more of the mathematical details, whereas undergraduate understanding will be largely via computational reasoning.
  
\item \emph{How will graduate student assignments differ from those of the undergraduates in their nature or quantity?}

  See above for comments to this question regarding assignments. Midterms and exams will be graded more strictly with respect to correctness and clarity of exposition.

\item \emph{How will increased opportunities for independent study or for interaction with the instructor(s) be made available for graduate students?}

  Graduate student assignments will require additional independent reading/study, and the instructor will make time available specifically for graduate students during office hours. 
  
\item \emph{Discuss any synthesis experiences specifically for graduates?}

  A term project will be assigned to graduate students that will comprise 20\% of the course grade. This project will require students to synthesize and apply the course content to a research setting. The students will be expected to formulate a problem, collect or simulate data, model the data, quantify uncertainty, and communicate these findings to foster knowledge discovery.

\item \emph{Identify on the proposed syllabus the opportunities the graduates will have for work at a higher academic level.}

  The course syllabus discusses these aspects under the ``400/600 Students'' heading. 

\item \emph{Discuss how the work of graduate students will be evaluated differently from that of undergraduates by describing the criteria used in grading an undergraduate assignment versus those used in grading a graduate level assignment.}

  Midterms and exams will be graded more strictly with respect to conceptual understanding, correctness, and clarity of exposition. 

\item \emph{How will graduates leave the course feeling that they have obtained greater academic value?}

The graduate students will feel that they obtained greater academic value as a result of the following measures: (i) Exposure to deeper level of material in homework exercises and exams, (ii) different grading criteria that emphasize the research component of the class, (iii) additional opportunities to interact with the instructor, and (iv) completion of a term project to apply and synthesize the course content.
  
\end{enumerate}

\end{document}