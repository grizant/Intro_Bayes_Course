\documentclass[11pt,article,landscape]{memoir}
% Copyright (C) 2013 Andrew Gainer-Dewar <andrew.gainer.dewar@gmail.com>
% This file may be distributed and/or modified under the
% conditions of the LaTeX Project Public License, either
% version 1.2 of this license or (at your option) any later
% version. The latest version of this license is in:
% http://www.latex-project.org/lppl.txt
% and version 1.2 or later is part of all distributions of[
% LaTeX version 1999/12/01 or later.

\usepackage{agd-rubric}

\rubriccourse{MATH 429/629 Introduction to Bayesian Statistics}
\rubricterm{Spring 2019}
\rubricthing{Homework 5}
\rubrictopprompt{Name}
%% \rubricbottomprompt{Total Score}

\begin{document}
\maketitle

{\large Please complete exercises Ch.8, \#2,4,5,C1-C6 in Bolstad \& Curran. I selected these problems to give deliberate practice towards mastering learning outcome 3 --- \emph{students will conduct Bayesian inference for parameters of discrete and continuous random variables}. All exercises provide practice quantifying uncertainty in the estimation of a binomial proportion. I'm also focusing on the computer exercises as the course will become more and more computational as we dig into more complicated settings. For the exercises that ask you to plot two curves on the same graph, side-by-side plots (created with \textsc{binobp} or other \textsc{Bolstad} function) is acceptable. The problems emphasize prior selection and implications. You have the option of calculating these quantities by hand (using formula/tables) or by writing R code --- just be sure to justify your work well.}

% The argument determines the number of score buckets
\begin{rubrictable}{4}{0}
  % Add your \rubricdesc items in increasing order!

  %% Category Description
  \rubriccat{Point value label}{
    \rubricdesc{Needs Improvement}{
      
    }
    \rubricdesc{Approaching standards}{
      
    }
    \rubricdesc{Meets standards}{
      
    }
    \rubricdesc{Exceeds standards}{
      
    }
    \rubrictimes{NA}
  }
  
    %% Category 1
    \rubriccat{1.~Explanation and justification}{
    \rubricdesc{}{
      Explanation is difficult to understand and is missing several components OR was not included.
    }
    \rubricdesc{}{
      Explanation is a little difficult to understand, but includes critical components.
    }
    \rubricdesc{}{
      Explanation is clear.
    }
    \rubricdesc{}{
      Explanation is detailed and clear yet concise.
    }
    \rubrictimes{2}
  }
  
  %% Category 2
    \rubriccat{2.~Mathematical accuracy}{
    \rubricdesc{}{
      More than ~25\% of computations/statements have errors in selected problems. 
    }
    \rubricdesc{}{
      Most ~(75\%-89\%) of computations/statements are free of errors in selected problems.
    }
    \rubricdesc{}{
      Almost all ~(90\%-99\%) of computations/statements are free of errors in selected problems.
    }
    \rubricdesc{}{
      All of computations/statements are free of errors in selected problems.
    }    \rubrictimes{2}
  }

  %% Category 3
    \rubriccat{3.~Completion}{
    \rubricdesc{}{
      Several problems are not completed. 
    }
    \rubricdesc{}{
      All but two of the problems are completed.
    }
    \rubricdesc{}{
      One problem not completed.
    }
    \rubricdesc{}{
      All problems addressed.
    }
    \rubrictimes{1}
  }

    %% Category 4
    \rubriccat{4.~Neatness and organization}{
    \rubricdesc{}{
      The work appears sloppy and unorganized. It is hard to know what information goes together.
    }
    \rubricdesc{}{
      The work is presented in an organized fashion but may be hard to read at times.
    }
    \rubricdesc{}{
      The work is presented in a neat and organized fashion that is usually easy to read.
    }
    \rubricdesc{}{
      The work is presented in a neat, clear, organized fashion that is easy to read.
    }
    \rubrictimes{1}
  }

    %% Category 5
    \rubriccat{5.~Mathematical terminology and notation}{
    \rubricdesc{}{
      There is little use, or a lot of inappropriate use, of terminology and notation.
    }
    \rubricdesc{}{
      Correct terminology and notation are used, but it is sometimes not easy to understand what was done.
    }
    \rubricdesc{}{
      Correct terminology and notation are usually used, making it fairly easy to understand what was done.
    }
    \rubricdesc{}{
      Correct terminology and notation are always used, making it easy to understand what was done.
    }
    \rubrictimes{1}
  }

\end{rubrictable}

\huge Total Score (out of 20 points, 21 points possible): \hrulefill
\end{document}