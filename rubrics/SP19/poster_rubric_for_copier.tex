%A simple LaTeX syllabus template by Luke Smith.
%I believe this is a variation of a template I got by Mike Hammond...?

%\documentclass[11pt,twocolumn]{article}
\documentclass[11pt,onecolumn]{article}
\usepackage{hyperref}
\usepackage[margin=0.75in]{geometry}
\usepackage{titlesec}
\usepackage{longtable}
\usepackage{gb4e}
%\usepackage[margin=.75in]{geometry}
\usepackage{textcomp}
\pagenumbering{gobble}
%% \usepackage[utf8]{inputenc}
%% \usepackage[english]{babel}
%% \usepackage[usenames, dvipsnames]{color}

%paragraph formatting
\setlength{\parindent}{0pt}
\setlength{\parskip}{3pt}

\usepackage{url}

\hypersetup{
    colorlinks=true,
    linkcolor=blue,
    filecolor=magenta,      
    urlcolor=blue,
}

\titlespacing\subsection{0in}{\parskip}{\parskip}
\titlespacing\section{0in}{\parskip}{\parskip}

%Just fill these in to fill in the basic syllabus information.
\newcommand{\coursename}{Introduction to Bayesian Statistics\\Poster competency matrix}
\newcommand{\semester}{Spring 2019}
\newcommand{\roomnumb}{AB 206}
\newcommand{\classtimes}{Mon,Wed 1:00pm - 2:15pm}
\newcommand{\myname}{A.~Grant Schissler}
%\newcommand{\myemail}{\href{mailto:aschissler@unr.edu}{aschissler@unr.edu}}
\newcommand{\myemail}{aschissler@unr.edu}
\newcommand{\office}{DMSC 224}
\newcommand{\officehours}{Tue 2:30pm-3:30pm, Wed 1:30pm-2:30pm, or by appointment}
% \newcommand{\university}{The University of Nevada, Reno}
\renewcommand{\familydefault}{\sfdefault}

\title{\textbf{\coursename}}
% \author{{\university}---{\semester}---{\roomnumb}---{\classtimes}}
\author{{\semester}---{\roomnumb}---{\classtimes}}
\date{}


\begin{document}
%% \maketitle

%% \vspace{-0.25in}
%% \noindent\makebox[\linewidth]{\rule{\textwidth}{1pt}}

%% \begin{center}
%% \begin{tabular}{llll}
%% \textbf{Instructor}:&\myname & \textbf{Contact}:&\href{mailto:\myemail}{\myemail}, 775-784-4661 (office)\\
%% \textbf{Office}:&\office & \textbf{Hours}:&\officehours\\
%% \end{tabular}
%% \end{center}

Name (Student evaluator): 

Project members: 

Total score (out of 12): 

\begin{table}[htb]
  \centering
  \begin{tabular}{|p{4.5cm}|p{4.2cm}|p{3.75cm}|p{3.5cm}|}
    \hline
    \bf Critical task & \bf Needs improvement (1) & \bf Basic (2) & \bf Surpassed (3) \\
    \hline
    \hline
   %%  \textbf{Computation.} Perform computations necessary for the data analysis & Computations contain errors and extraneous code & Computations correct but contain extraneous code/unnecessary code & Computations correct, clear, and properly labeled\\
   %%  \hline
    \textbf{Analysis.} Choose and carry out analysis appropriate for data and context. MUST USE BAYESIAN STATISTICS! & Choice of analysis is overly simplistic, irrelevant, inappropriate for the data, or missing key component(s)& Analysis appropriate, but incomplete and important features and assumptions not made explicit & Analysis appropriate, complete, advanced, relevant, and informative\\
    \hline
    \textbf{Synthesis.} Identify key features of the analysis, and interpret results in context & Conclusions are missing, incorrect, or unsupported & Conclusions are reasonable, but partially correct or partially complete & Relevant conclusions explicitly connected to analysis and context\\
    \hline
    \textbf{Visual.} Communicate findings graphically clearly, precisely, and concisely & Inappropriate choice of plots; poorly labeled plots; plots missing & Plots convey information correctly but lack context for interpretation & Plots convey information correctly with adequate and appropriate connections\\
    \hline
    \textbf{Written.} Communicate findings in writing clearly, precisely, and concisely & Explanation is illogical, incorrect, or incoherent & Explanation is partially correct but incomplete or unconvincing & Explanation is correct, complete, and convincing.\\
    \hline
  \end{tabular}
\end{table}

%% Second rubric
\vskip1in

Name (Student evaluator): 

Project members: 

Total score (out of 12): 

\begin{table}[htb]
  \centering
  \begin{tabular}{|p{4.5cm}|p{4.2cm}|p{3.75cm}|p{3.5cm}|}
    \hline
    \bf Critical task & \bf Needs improvement (1) & \bf Basic (2) & \bf Surpassed (3) \\
    \hline
    \hline
   %%  \textbf{Computation.} Perform computations necessary for the data analysis & Computations contain errors and extraneous code & Computations correct but contain extraneous code/unnecessary code & Computations correct, clear, and properly labeled\\
   %%  \hline
    \textbf{Analysis.} Choose and carry out analysis appropriate for data and context. MUST USE BAYESIAN STATISTICS! & Choice of analysis is overly simplistic, irrelevant, inappropriate for the data, or missing key component(s)& Analysis appropriate, but incomplete and important features and assumptions not made explicit & Analysis appropriate, complete, advanced, relevant, and informative\\
    \hline
    \textbf{Synthesis.} Identify key features of the analysis, and interpret results in context & Conclusions are missing, incorrect, or unsupported & Conclusions are reasonable, but partially correct or partially complete & Relevant conclusions explicitly connected to analysis and context\\
    \hline
    \textbf{Visual.} Communicate findings graphically clearly, precisely, and concisely & Inappropriate choice of plots; poorly labeled plots; plots missing & Plots convey information correctly but lack context for interpretation & Plots convey information correctly with adequate and appropriate connections\\
    \hline
    \textbf{Written.} Communicate findings in writing clearly, precisely, and concisely & Explanation is illogical, incorrect, or incoherent & Explanation is partially correct but incomplete or unconvincing & Explanation is correct, complete, and convincing.\\
    \hline
  \end{tabular}
\end{table}


\end{document}