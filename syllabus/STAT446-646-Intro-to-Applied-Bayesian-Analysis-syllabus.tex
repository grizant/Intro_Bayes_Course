%A simple LaTeX syllabus template by Luke Smith.
%I believe this is a variation of a template I got by Mike Hammond...?

%\documentclass[11pt,twocolumn]{article}
\documentclass[11pt,onecolumn]{article}
\usepackage{hyperref,color}
\usepackage[margin=1in]{geometry}
\usepackage{titlesec}
\usepackage{longtable}
\usepackage{gb4e}
%\usepackage[margin=.75in]{geometry}
\usepackage{textcomp}
\pagenumbering{gobble}

%paragraph formatting
\setlength{\parindent}{0pt}
\setlength{\parskip}{7pt}

\hypersetup{
    colorlinks=true,
    linkcolor=blue,
    filecolor=magenta,      
    urlcolor=blue,
}

\titlespacing\subsection{0in}{\parskip}{\parskip}
\titlespacing\section{0in}{\parskip}{\parskip}

%Just fill these in to fill in the basic syllabus information.
\newcommand{\coursename}{Introduction to Applied Bayesian Analysis - STAT 446/646}
\newcommand{\semester}{Spring 2019}
\newcommand{\roomnumb}{DMSC 106}
\newcommand{\classtimes}{MWF 9:00AM - 10:0AM}
\newcommand{\myname}{A.~Grant Schissler}
% \newcommand{\myemail}{\href{mailto:aschissler@unr.edu}{aschissler@unr.edu}}
\newcommand{\myemail}{aschissler@unr.edu}
\newcommand{\office}{DMSC 224}
\newcommand{\officehours}{TBD, or by appointment}
% \newcommand{\university}{The University of Nevada, Reno}
\renewcommand{\familydefault}{\sfdefault}

\title{\textbf{\coursename}}
% \author{{\university}---{\semester}---{\roomnumb}---{\classtimes}}
\author{{\semester}---{\roomnumb}---{\classtimes}}
\date{}


\begin{document}
\maketitle

\noindent\makebox[\linewidth]{\rule{\textwidth}{1pt}}

\begin{center}
\begin{tabular}{llll}
\textbf{Instructor}:&\myname & \textbf{Contact}:&\href{mailto:\myemail}{\myemail}, 775-784-4661 (office)\\
\textbf{Office}:&\office & \textbf{Hours}:&\officehours\\
\end{tabular}
\end{center}

Bayesian data analysis offers a flexible and powerful approach to modeling data and scientific inquiry. The target audience has had some linear algebra, calculus, and computer programming along one or two statistics classes. This course serves as an introduction to modern Bayesian modeling and computation, with an emphasis on generative data modeling using probability distributions, Monte Carlo simulation, and model evaluation.

The approach to instruction is largely computational in both the readings and assigned class activities. Further mathematical depth will be discussed but not emphasized. Information theory and entropy will be introduced as motivation for model construction and evaluation. Further topics include the basics of Bayesian analysis, Markov Chain Monte Carlo (MCMC), regression through multilevel models, measurement error, missing data, and Gaussian process models for spatial and network autocorrelation. The main goal is empower learners to confidently perform and communicate a challenging Bayesian analysis using state-of-the-art statistical computing and modeling techniques. 

\section*{Catalog Description}
Introduction to Bayesian data analysis; statistical computing; simulation techniques; Markov Chain Monte Carlo; information theory; advanced statistical methods; generalized multi-level models.

\section*{Course Pre-requisites}
STAT 352 or STAT 467 or STAT 667 or with instructor approval.

\section*{Student Learning Outcomes}
\begin{enumerate}
\item Students will be able to demonstrate understanding of the concepts that underly modern methods of Bayesian analysis, and critically assess the assumptions associated with different statistical models.
\item Students will be able to interpret and discuss the results of Bayesian analysis analyses.
\item Students will be able to perform a modern Bayesian analysis using professional statistical packages.
\end{enumerate}

\section*{Required text}
\emph{Statistical Rethinking}, by Richard McElreath. \\
Textbook website:~\url{http://xcelab.net/rm/statistical-rethinking/}

\section*{Course Topics}
Below is a tentative list topics for the course, and the order they will be covered. See the course website for a more detailed list of topics, and updated schedule.
\begin{enumerate}
\item Basics of Bayesian analysis
\item Bayesian computation, sampling the posterior
\item Linear Models
\item Multivariate Linear Models
\item Overfitting and Model Comparison
\item Interactions in models
\item Markov chain Monte Carlo Estimation
\item Entropy and the Generalized Linear Model
\item Counting and Classification
\item Mixtures distributions
\item Generalized Multilevel Models
\item Multivariate analysis (Covariance/Correlation)
\item Missing Data
\end{enumerate}

\section*{Assignments}
Exercises will be assigned approximately weekly. You are encouraged to discuss  assignments between each other and with instructor. However, the works must be written individually.

\section*{Midterms}
There will be two midterms covering approximately the first and second third of the semester.

\section*{Final exam}
A comprehensive final exam will be held during the time scheduled in the UNR catalog.

\section*{400/600 Students}
Assignments and exams for students enrolled at the 400 level and 600 level will differ. Additionally, students enrolled at the 600 level will be held to a higher academic standard with respect to grading.

\section*{Makeup, Late Policy}
Late assignments, exams, and projects will not be graded. There will be no early or make-up exams. However, if you need to miss an exam due to participation in official university activities, you must make arrangements with the instructor at least two weeks prior to the exam in question. Since the late policy is rather strict, I will drop your lowest two grades in the ``Assignments'' category as a safety factor for emergencies.

\section*{Grading}
The final grades will be determined using the following percentages:

\begin{center}
\begin{tabular}{cc}
\begin{tabular}{l|l}	%For grade items (quizzes, homework, etc.)
Item&Percent\\\hline\hline
  Assignments&55\%\\
  Midterm Exams&20\%\\
  Final Exam&25\%\\
\end{tabular}
&
\begin{tabular}{ll}
A&90-100\\
B&80-89\\
C&70-79\\
D&60-69\\
F&59 or below
\end{tabular}
\end{tabular}
\end{center}

The instructor reserves the right to deviate from the above percentages in special cases, including borderline cases (generally this could be +/- 3\% points) may be given a + or − within the above intervals or increasing the letter grade.

\section*{Diversity Statement}
The University of Nevada, Reno is committed to providing a safe learning and work environment for all. If you believe you have experienced discrimination, sexual harassment, sexual assault, domestic/dating violence, or stalking, whether on or off campus, or need information related to immigration concerns, please contact the University’s Equal Opportunity \& Title IX Office at (775) 784-1547. Resources and interim measures are available to assist you. For more information, please visit \url{http:www.unr.edu/equal-opportunity-title-ix}.

\section*{Disability Statement}
The Department of Mathematics and Statistics supports providing equal access for students with disabilities. Any student with a disability needing academic adjustments or accommodations is requested to speak with me or the Disability Resource Center (PSAC 230, \url{http:www.unr.edu/drc}) as soon as possible to arrange for appropriate accommodations.

\section*{Academic Conduct}
No laptops, cell phones, mp3 players, or other electronics are to be used for personal reasons in class. If you are being disruptive during class you will be asked to leave. Disruptions in this context include inadequate participation. You must come to class on time and stay until the end of lecture. Tardy students will not be admitted to class. Please visit \url{http:www.unr.edu/student-conduct} for our official student code of conduct.

\section*{Academic Success Services}
A common habit among successful students is to seek help outside of the classroom. Your student fees cover use of the Math Center (784-4433 or \url{http:www.unr.edu/mathcenter}), Tutoring Center (784-6801 or \url{http:www.unr.edu/tutoring-center}), and University Writing Center (784-6030 or \url{http:www.unr.edu/writing-center}). These centers support your classroom learning; it is your responsibility to take advantage of their services.

\section*{University Recording Policy}
Surreptitious or covert videotaping of class or unauthorized audio recording of class is prohibited by law and by Board of Regents policy. This class may be videotaped or audio recorded only with the written permission of the instructor. In order to accommodate students with disabilities, some students may have been given permission to record class lectures and discussions. Therefore, students should understand that their comments during class may be recorded.

\section*{Academic Dishonesty}
Cheating, plagiarism, or otherwise obtaining grades under false pretenses constitutes academic dishonesty according to the code of this university. Academic dishonesty will not be tolerated and penalties can include canceling a student’s enrollment without a grade or giving an F for the assignment or for the entire course. For more details, see the University of Nevada, Reno general catalog.

\end{document}